\documentclass[12pt]{article}
\usepackage[utf8]{inputenc}
\usepackage[T1]{fontenc}
\usepackage[scale=.9]{courierten}
\usepackage{step}\usepackage[notext]{stix}
\normalfont\parindent=1.5em
\usepackage[letterpaper]{geometry}
\usepackage[greek,main=english]{babel}
\usepackage{hologo}
\usepackage[expansion=false,protrusion,babel]{microtype}
\DeclareMicrotypeAlias{\rmdefault}{ptm}
\usepackage{footmisc}
\makeatletter
\renewcommand\@makefnmark{\hbox{\@textsuperscript{\normalfont\lining{\@thefnmark}}}}
\DeclareRobustCommand*\textsuperscript[1]{\@textsuperscript{\lining{#1}}}
\DeclareRobustCommand*\textsubscript[1]{\@textsubscript{\lining{#1}}}
\makeatother



\usepackage{booktabs}
\title{The STEP fonts}
\date{Version 2.0.4a\\\today}
\author{Daniel~Benjamin Miller\thanks{\texttt{dbmiller@dbmiller.org}}}
\begin{document}
\maketitle
The STEP fonts are a free Times-like (i.e., Times replacement) font family, implementing a design first created for \textit{The Times} of London in 1932. These fonts are meant to be compatible in design with Adobe's digitization of Linotype Times, which is commonly used in desktop publishing.

\section{Usage}
To use the STEP fonts, load the \LaTeX\ package \texttt{step}, optionally with \texttt{stix} for math.
\begin{verbatim}
\usepackage{step}\usepackage[notext]{stix}
\end{verbatim}

If you are using \hologo{LuaLaTeX} or \hologo{XeLaTeX}, you may use the Unicode version of STEP with \texttt{fontspec}, along with XITS for math (using \texttt{unicode-math}).
\begin{verbatim}
\usepackage{fontspec,unicode-math}\setmainfont[]{STEP}
\setmathfont[StylisticSet=0]{XITS Math}
\setmathfont[range={\mathcal,\mathbfcal},StylisticSet=1]{XITS Math}
\end{verbatim}
When using \texttt{fontspec}, all font options are handled according to that package's functions. See its documentation for details.
\subsection{Package options}
\begin{tabular}{@{} ll @{}}
\toprule
Package Option & Feature \\
\midrule
\texttt{lining} & Use lining figures (default) \lining{1234567890}\\
\texttt{oldstyle} & Use old-style figures \oldstyle{1234567890}\\
\bottomrule
\end{tabular}
\subsection{Text Commands}
\begin{tabular}{@{} ll @{}}
\toprule
Text Command & Feature \\
\midrule
\texttt{\textbackslash lining\{\}} & Use lining figures \lining{1234567890}\\
\texttt{\textbackslash oldstyle\{\}} & Use old-style figures \oldstyle{1234567890}\\
\texttt{\textbackslash textsc\{\}} & Use small caps (regular font only) \textsc{abcde\$\&}\\
\texttt{\textbackslash textsu\{\}} & Use superior figures \textsu{1234567890}\\
\texttt{\textbackslash textin\{\}} & Use inferior figures \textin{1234567890}\\
\bottomrule
\end{tabular}
\subsection{Greek text fonts}
For a Greek font which is better suited for text (and which includes bold, italic and bold italic as well as regular), I recommend using the fonts from \texttt{txfontsb} or \texttt{tempora}, which provide good support for monotonic and polytonic Greek, or the beta version of STEP Greek, which only includes a regular upright font at writing (with full polytonic support). You can access these by adding the following to your preamble (choosing one of \texttt{txrc}, \texttt{Tempora-TLF}, or \texttt{STEPGreekTest-TLF}):
\begin{verbatim}
\usepackage{substitutefont}
\substitutefont{LGR}{\rmdefault}{<fontname>}
\end{verbatim}

If you are using the OpenType version of STEP, I recommend you use the Babel function for language-specific OpenType fonts (covered in detail in the \texttt{babel} documentation). For a test version of the new STEP Greek font, available on GitHub,\footnote{\texttt{https://github.com/dbenjaminmiller/STEPGreekTest}} use:

\begin{verbatim}
\babelfont[greek]{rm}[]{STEPGreekTest}
\end{verbatim}

\subsection{Figures in footnotes}
The superior figures included in STEP may not be ideal for use in footnotes. For example, \textsu{25} and \textsu{15} print awkwardly. So I recommend you not use superior figures, but instead normal lining figures for footnotes. If you have set your numbers to be old-style, then, by \LaTeX's defaults, the old-style numbers will be used for footnotes as well. To use lining figures in footnote markers, as well as superscript and subscript text, add the following to your preamble:
\begin{verbatim}
\usepackage{footmisc}
\makeatletter
\renewcommand\@makefnmark{
\hbox{\@textsuperscript{\normalfont\lining{\@thefnmark}}}}
\DeclareRobustCommand*\textsuperscript[1]{
\@textsuperscript{\lining{#1}}}
\DeclareRobustCommand*\textsubscript[1]{
\@textsubscript{\lining{#1}}}
\makeatother
\end{verbatim}
\section{Contributing}
Contributions to the fonts and their \LaTeX\ support may be sent as pull requests to the project's GitHub repository.\footnote{\texttt{https://github.com/dbenjaminmiller/step}}
\section{Licensing}
STEP is licensed under the Open Font License (OFL) version 1.1; see \texttt{STEP-COPYING} for details.
\end{document}