\documentclass[12pt]{article}
\usepackage[utf8]{inputenc}
\usepackage[T1]{fontenc}
\usepackage{times,newtxmath}
\usepackage[scale=.9]{courierten}
\usepackage{step}
\usepackage[letterpaper]{geometry}
\usepackage[greek,main=english]{babel}
\usepackage{hologo}
\usepackage[expansion=false,protrusion,babel]{microtype}
\DeclareMicrotypeAlias{\rmdefault}{ptm}
\usepackage{footmisc}
\makeatletter
\renewcommand\@makefnmark{\hbox{\@textsuperscript{\normalfont\lining{\@thefnmark}}}}
\DeclareRobustCommand*\textsuperscript[1]{\@textsuperscript{\lining{#1}}}
\DeclareRobustCommand*\textsubscript[1]{\@textsubscript{\lining{#1}}}
\makeatother



\usepackage{booktabs}
\title{The STEP fonts}
\date{Version 2.0.4\\\today}
\author{Daniel~Benjamin Miller\thanks{\texttt{dbmiller@dbmiller.org}}}
\begin{document}
\maketitle
The STEP fonts are a free Times-like (i.e., Times replacement) font family, implementing a design first created for \textit{The Times} of London in 1932. These fonts are meant to be compatible in design with Adobe's digitization of Linotype Times, which is commonly used in desktop publishing.

\section{Usage}
To use the STEP fonts, load the \LaTeX\ package \texttt{step}.
\begin{verbatim}
\usepackage{step}
\end{verbatim}
Do remember that it is usually wise to load \texttt{fontenc} with the appropriate encoding(s).

For mathematics, it is recommended to add \texttt{\textbackslash usepackage\{times,newtxmath\}} (note that this must be loaded \emph{before} STEP). See the documentation for \texttt{newtxmath} to find out more about available options.

If you are using \hologo{LuaLaTeX} or \hologo{XeLaTeX}, you may use the Unicode version of STEP with \texttt{fontspec}\footnote{\texttt{https://ctan.org/pkg/fontspec}}:
\begin{verbatim}
\usepackage{fontspec}\setmainfont{STEP}
\end{verbatim}
When using \texttt{fontspec}, all font options are handled according to that package's functions. See its documentation for details. I recommend that even if using OpenType for the body text, that you use the math fonts in the conventional way (mentioned above), but, if you wish, you may use XITS Math.

\subsection{Package options}
\begin{tabular}{@{} ll @{}}
\toprule
Package Option & Feature \\
\midrule
\texttt{lining} & Use lining figures (default) \lining{1234567890}\\
\texttt{oldstyle} & Use old-style figures \oldstyle{1234567890}\\
\bottomrule
\end{tabular}
\subsection{Text Commands}
\begin{tabular}{@{} ll @{}}
\toprule
Text Command & Feature \\
\midrule
\texttt{\textbackslash lining\{\}} & Use lining figures \lining{1234567890}\\
\texttt{\textbackslash oldstyle\{\}} & Use old-style figures \oldstyle{1234567890}\\
\texttt{\textbackslash textsc\{\}} & Use small caps (regular font only) \textsc{abcde\$\&}\\
\texttt{\textbackslash textsu\{\}} & Use superior figures \textsu{1234567890}\\
\texttt{\textbackslash textin\{\}} & Use inferior figures \textin{1234567890}\\
\bottomrule
\end{tabular}
\subsection{Greek text fonts}
For a Greek font which is better suited for text (and which includes bold, italic and bold italic as well as regular), I recommend using the fonts from \texttt{txfontsb} or \texttt{tempora}, which provide good support for monotonic and polytonic Greek. You can access these by adding the following to your preamble (choosing one of \texttt{txrc} or \texttt{Tempora-TLF}):
\begin{verbatim}
\usepackage{substitutefont}
\substitutefont{LGR}{\rmdefault}{<txrc> or <Tempora-TLF>}
\end{verbatim}

If you are using the OpenType version of STEP, I recommend you use the Babel function for language-specific OpenType fonts (covered in detail in the \texttt{babel}\footnote{\texttt{https://ctan.org/pkg/babel}} documentation). For a test version of the new STEP Greek font (only available in upright form at writing), see the test repository on GitHub.\footnote{\texttt{https://github.com/dbenjaminmiller/STEPGreekTest}}

\begin{verbatim}
\babelfont[greek]{rm}[]{STEPGreekTest}
\end{verbatim}

\subsection{Figures in footnotes}
The superior figures included in STEP may not be ideal for use in footnotes. For example, \textsu{25} and \textsu{15} print awkwardly. So I recommend you not use superior figures, but instead normal lining figures for footnotes. If you have set your numbers to be old-style, then, by \LaTeX's defaults, the old-style numbers will be used for footnotes as well. To use lining figures in footnote markers, as well as superscript and subscript text, add the following to your preamble:
\begin{verbatim}
\usepackage{footmisc}
\makeatletter
\renewcommand\@makefnmark{
\hbox{\@textsuperscript{\normalfont\lining{\@thefnmark}}}}
\DeclareRobustCommand*\textsuperscript[1]{
\@textsuperscript{\lining{#1}}}
\DeclareRobustCommand*\textsubscript[1]{
\@textsubscript{\lining{#1}}}
\makeatother
\end{verbatim}
\section{Contributing}
Contributions to the fonts and their \LaTeX\ support may be sent as pull requests to the project's GitHub repository.\footnote{\texttt{https://github.com/dbenjaminmiller/step}}
\section{Licensing}
STEP is licensed under the Open Font License (OFL) version 1.1; see \texttt{STEP-COPYING} for details.
\end{document}